\documentclass[11pt,]{article}
\usepackage[left=1in,top=1in,right=1in,bottom=1in]{geometry}
\newcommand*{\authorfont}{\fontfamily{lmr}\selectfont}
\usepackage{lmodern}

% added to ensure new version of pandoc doesn't complain

  \usepackage[T1]{fontenc}
  \usepackage[utf8]{inputenc}


\usepackage{abstract}
\renewcommand{\abstractname}{}    % clear the title
\renewcommand{\absnamepos}{empty} % originally center
\setlength{\parindent}{0pt}
\setlength{\parskip}{6pt plus 2pt minus 1pt}

\renewenvironment{abstract}
 {{%
    \setlength{\leftmargin}{0mm}
    \setlength{\rightmargin}{\leftmargin}%
  }%
  \relax}
 {\endlist}

\makeatletter
\def\@maketitle{%
  \newpage
%  \null
%  \vskip 2em%
%  \begin{center}%
  \let \footnote \thanks
    {\fontsize{18}{20}\selectfont\raggedright  \setlength{\parindent}{0pt} \@title \par}%
}
%\fi
\makeatother


%
% header and footer
%

\usepackage{fancyhdr}

\fancypagestyle{eisvogel-header-footer}{
  \fancyhead{}
  \fancyfoot{}
  \lhead[]{}
  \chead[]{}
  \rhead[]{}
  \lfoot[\thepage]{}
  \cfoot[]{}
  \rfoot[]{\thepage}
  \renewcommand{\headrulewidth}{0.4pt}
  \renewcommand{\footrulewidth}{0.4pt}
}
\pagestyle{eisvogel-header-footer}



\setcounter{secnumdepth}{0}






\author{}

\date{}






\newtheorem{hypothesis}{Hypothesis}
\usepackage{setspace}


% set default figure placement to htbp
\makeatletter
\def\fps@figure{htbp}
\makeatother

\usepackage{booktabs}
\usepackage{longtable}
\usepackage{array}
\usepackage{multirow}
\usepackage{wrapfig}
\usepackage{float}
\usepackage{colortbl}
\usepackage{pdflscape}
\usepackage{tabu}
\usepackage{threeparttable}
\usepackage{threeparttablex}
\usepackage[normalem]{ulem}
\usepackage{makecell}
\usepackage{relsize,etoolbox}
\usepackage{tcolorbox}
\definecolor{block-gray}{gray}{0.95}
\usepackage{amsmath}
\newcommand\qed{\hfill\rule{1em}{1em}}
\newtcolorbox{blockquote}{colback=block-gray,grow to right by=-8mm,grow to left by=-8mm,boxrule=0pt,boxsep=0pt}
\AtBeginEnvironment{blockquote}{\small}

% move the hyperref stuff down here, after header-includes, to allow for - \usepackage{hyperref}

\makeatletter
\@ifpackageloaded{hyperref}{}{%
\ifxetex
  \PassOptionsToPackage{hyphens}{url}\usepackage[setpagesize=false, % page size defined by xetex
              unicode=false, % unicode breaks when used with xetex
              xetex]{hyperref}
\else
  \PassOptionsToPackage{hyphens}{url}\usepackage[draft,unicode=true]{hyperref}
\fi
}

\@ifpackageloaded{color}{
    \PassOptionsToPackage{usenames,dvipsnames}{color}
}{%
    \usepackage[usenames,dvipsnames]{color}
}
\makeatother
\hypersetup{breaklinks=true,
            bookmarks=true,
            pdfauthor={},
             pdfkeywords = {},
            pdftitle={},
            colorlinks=true,
            citecolor=blue,
            urlcolor=blue,
            linkcolor=magenta,
            pdfborder={0 0 0}}
\urlstyle{same}  % don't use monospace font for urls

% Add an option for endnotes. -----


% add tightlist ----------
\providecommand{\tightlist}{%
\setlength{\itemsep}{0pt}\setlength{\parskip}{0pt}}

% add some other packages ----------

% \usepackage{multicol}
% This should regulate where figures float
% See: https://tex.stackexchange.com/questions/2275/keeping-tables-figures-close-to-where-they-are-mentioned
\usepackage[section]{placeins}


\begin{document}

% \pagenumbering{arabic}% resets `page` counter to 1
%




\vskip -8.5pt


 % removetitleabstract

\noindent \singlespacing

\hypertarget{manuscript-number-ismej-20-01631a}{%
\subsection{Manuscript number:
ISMEJ-20-01631A}\label{manuscript-number-ismej-20-01631a}}

\hypertarget{title-competitive-hierarchies-antibiosis-and-the-distribution-of-bacterial-life-history-traits-in-a-microbiome}{%
\subsubsection{Title: ``Competitive hierarchies, antibiosis, and the
distribution of bacterial life history traits in a
microbiome''}\label{title-competitive-hierarchies-antibiosis-and-the-distribution-of-bacterial-life-history-traits-in-a-microbiome}}

\begin{blockquote}
Dear Professor Whiteman,

Thank you for submitting your manuscript to The ISME Journal, which has
been reviewed by several experts in the field. We are enclosing their
comments below. Although the reviewers think that the work is of
potential interest, substantial concerns have also been raised that will
preclude publication of the paper in its present form. We are willing to
consider a revised version of the manuscript provided that you satisfy
the criticisms raised by the reviewers and/or the editors. Generally, a
revised manuscript should be re-submitted within 1 month. Additional
time for revision may be granted to the authors upon request; please
inform the editorial office as soon as possible if you anticipate the
need for such an extension.

Should you decide to re-submit, please include a cover letter and a
point-by-point response to the referees' and editors' comments. All
comments need to be addressed and if you do not agree with some of them,
please specify the reasons.

In addition to a clean revised manuscript, please also supply a track
change PDF version of your revised manuscript so the editor and
reviewers can readily assess the revisions that you have made. Please
use the ``Print to PDF'' option for generating a PDF file in Word,
rather than the ``Save to PDF'' option, to preserve tracked changes in
the exported PDF file.

Please note that The ISME Journal will only publish manuscripts if
authors agree to make all data that cannot be published in the journal
itself freely available in one of the public databases. Accession codes
must be provided. More information about this can be found in the author
instructions.

The ISME Journal is committed to improving transparency in authorship.
As part of our efforts in this direction, we are now requesting that all
authors identified as ``corresponding author'' create and link their
Open Researcher and Contributor Identifier (ORCID) with their account on
the Manuscript Tracking System prior to acceptance. ORCID helps the
scientific community achieve unambiguous attribution of all scholarly
contributions. You can create and link your ORCID from the home page of
the Manuscript Tracking System by clicking on ``Modify my Springer
Nature account'' and following the instructions in the link below.
Please also inform all co-authors that they can add their ORCIDs to
their accounts and that they must do so prior to acceptance.

https://www.springernature.com/gp/researchers/orcid/orcid-for-nature-research

For more information please visit http://www.springernature.com/orcid

If you experience problems in linking your ORCID, please contact the
Platform Support Helpdesk.

The revised manuscript should be re-submitted via our website:

https://mts-isme.nature.com/cgi-bin/main.plex?el=A7Br3GzX2A3bwt2I5A9ftdyR9YWH3hvb6Z3icaWPOgZ

Please feel free to contact our office if you have questions about this
process.

Sincerely, George Kowalchuk Senior Editor The ISME Journal
\end{blockquote}

\begin{blockquote}
Editor in Chief (JDN) comments:
\end{blockquote}

\begin{blockquote}
Figure 1/2 should have all bootstrap values rounded to the same \%
(i.e., all without decimal places for consistency). Figure 2C should
have all numbers reported to same decimal places.
\end{blockquote}

These changes have been made.

\begin{blockquote}
Senior Editor:

Although the reviewers found the results presented in your manuscript
were of potential interest, they also expressed a number of important
issues that would have to be addressed in revision for your manuscript
to be considered further. Important issues include the clarity and
presentation of the introduction and methods, as well as how results are
extrapoled to more complex systems.
\end{blockquote}

We thank the Senior Editor and Editor in Chief for evaluating the
manuscript and for the opportunity to resubmit.

\begin{blockquote}
Referee \#1 (Comments to the Author):

It is interesting that the authors connected different traits of
bacteria with competitiveness by measuring pairwise interactions between
two clades of phyllosphere pseudomonas strains. In order to quantify
life history, the authors measured growth curve of each strain and
calculated lag phase, growth rate and maximum yield. After that,
correlations between life history and types of competitive interactions
were made. This work can give implications about how to predict species
interactions or community composition under competition. Moreover, this
can give some information about how to construct a microbial community
to defend pseudomonas invasion successfully. However, the manuscript
needs further revision. Some logic was missing especially in the
introduction part.

The introduction part needs a thorough revision. It is only roughly
written and lack of logic. For instance, the authors just wrote a tiny
paragraph to demonstrate why working with plants in the beginning of
introduction part. This has no connection with the 2nd paragraph. I
think the importance of this study is to correlate traits with
competitive interactions in the microbial community by using 40
phyllosphere bacteria. It is more necessary to develop more about
competitive interactions in the beginning (why it is so important and
what triats would possibly correlate with it) and introduce 40 model
strains in the end of the introduction. Background about phyllosphere
bacteria is also strange in line 55-60. Please rephrase this paragraph.
\end{blockquote}

We thank the reviewer for this important perspective on the
Introduction. We have entirely re-structured the flow of this section in
light of the reviewer's suggestions.

\begin{blockquote}
In line 48-49, the authors cited a reference about keystone taxa
actually. I am not sure this help introduce the background.
\end{blockquote}

The idea behind introducing the concept of keystone taxa was to
highlight how some species generate stronger indirect effects in
communities than others, and that often the magnitudes of such effects
are highly skewed towards one or a small number of taxa. This idea can
surely be explained more clearly, and we have strived to achieve this
additional clarity in our revision.

\begin{blockquote}
The authors wrote that there were various gaps highlighted in the
introduction. Please make them clear and try to organize them into one
or two knowledge gap or hypothesis.
\end{blockquote}

We have more clearly organized the various gaps addressed be our work.

\begin{blockquote}
There is no problem of incubating strains for 60h. However, the authors
used the final OD600 (at 60h I think) as the maximum yield (K). It is
not correct. Normally, bacteria will go into decline phase after
stationary phase, which is the time that bacteria obtain the maximum
yield. For instance, growth curve of strain 46A in Fig S1 had an obvious
decline phase. I suggest the authors to look at the data again. In
addition, the measurement on growth curve only have two replicates. I am
not sure this can guarantee quality of the data. The authors did not
clarify whether the two replicates (two wells) are in the same
microplate or not either. Please make sure there is no pseudo-replicate.
Please also make it clear how many replicates did the authors do in each
experiment/measurement in the method part.
\end{blockquote}

Thank you for bringing up this important point. We are now more clear in
the text (and our approach) in that we use \(\text{max}[OD_{600}]\) as
our measure of maximum yield, rather than the value recorded at
\(t = 60 h\). Regarding replicates, we are comfortable with two
replicates, as the variation among curves is remarkably small.

\begin{blockquote}
There were no panel f and g in Fig S3.
\end{blockquote}

This has been fixed.

\begin{blockquote}
Is there any explanation about Fig S4? It shows that it has connection
with Fig 1c, but apparently there is no Fig1c in the main text. It is
the same as Fig S6.
\end{blockquote}

The ambiguity in figure cross-referencing has been fixed.

\begin{blockquote}
The authors tried to link life history to competitiveness and defined
Pseudomonas strains into two types, fast and slow life history. Can
these types of traits refer to r and K strategists? It would be better
that the authors conclude which type of strategist correlate with which
typs of competitiveness.
\end{blockquote}

We can allude more clearly to the \(r\) versus \(K\) strategist
paradigm, which has a strong legacy in ecology but has not been applied
much in the microbial sphere. The idea breaks down in this context
because, while \emph{P. syringae} (Ps) achieves high fitness by
expressing a fast growth rate (i.e., high \(r\)), \emph{P. fluorescens}
(Pf) competitiveness correlates with a suite of traits including the
ability to inhibit neighbors via secretions. While low \(r\) is not
observed among top Pf competitors, the relatively shorter lag phase
among top competitors suggests that being fast (in this other dimension)
is a fitness-promoting trait. We have attempted to introduce the idea or
\(r\) and \(K\) selection whilst clarifying the limits of this analogy
in this context, in the revised Discussion.

\begin{blockquote}
There can be more explanation about the definition of competitive
hierarchy and the results about rock-paper-scissors game. Please see
Higgins et al., 2017 about co-occurrence of microbial species and
non-transitive interactions. It would be better if the authors can
provide results of three species interactions in vitro experiment to
compare with the inferred data.
\end{blockquote}

Unfortunately we are unable to perform additional experiments with trios
\emph{in vitro}. In light of this, we have undertaken a more thorough
discussion of how our results relate to the many ways that hierarchies
can be disrupted/modified in the context of non-transitivity (including
RPS). We have tried to limit the scope of this discussion because the
predicted influence of RPS dynamics in our system is minimal.

\begin{blockquote}
Referee \#2 (Comments to the Author):

This study attempts to dissect competitive interactions among
cooccurring bacterial strains within a leaf microbiome. The strains,
although fairly closely related, seem to have distinct life history
traits which makes it possible for the authors to look at correlations
between traits and competition and to infer potential life history
strategies. I really enjoyed reading this paper, it was a thorough dig
into some often overlooked and difficult to study aspects of
microbe-microbe interactions. I found some parts of the introduction and
methods hard to follow (specific comments below). I found the
conclusions about the potentially distinct competition strategies
between the two clades fascinating, but I was less convinced by the
inferred interactions between 3 strains (specific comments below).
\end{blockquote}

We thank the reviewer for this feedback, and we hope to have brought
some clarity in our revision. Additionally, we have attempted to more
clearly point out the limitations of our \emph{inferred} interaction
trios. In parallel, we now point out the value of such simulations based
on empirical data rather than, say, parameterizing a simulation from
first principles or some other baseline. It is our perspective that
these simulations are a unique feature of our work: rather than handing
off an empirical dataset to other researchers to await \emph{in silico}
study, we have included our own take.

\begin{blockquote}
Major points:

Spatial structure is important for several of the points here but little
background is given on the spatial structuring of endophytic bacterial
communities. I'm not familiar with how endophytic bacterial populations
are typically structured, spatially. Is there much work on this? My
understanding is that endophytic communities are often not very dense,
so how frequently are populations of three different strains interacting
in endophytic spaces? Particularly, how often would three strains all in
one genus be interacting, just by chance? Based on the work that these
isolates came from, how many Pseudomonas taxa were typically found in a
given sample (a leaf or whatnot)?
\end{blockquote}

This is a crucial point. While there are still several uncertainties in
the literature about these points, we will note that our work on
\emph{Cardamine cordifolia} has shown that single leaf discs can harbor
multiple distinct Ps and Pf isolates (Humphrey et al.~2014, 2020). The
observational conundrum here is, of course, that IF competitive
interactions are strong, we should not expect to observe many
co-infected leaf samples if our sampling scale is similar to the spatial
scale of competitive exclusion. The fact that we commonly observe
multiply-infected leaf samples (but not all the time) suggests either
that competition isn't really happening; or, perhaps more likely, that
we often sample leaf tissue extents that are larger than the general
scale of competitive exclusion in this system. The best answer we can
give here is that distinct strains of each Ps and Pf often co-occur in
single leaf samples in this system, which gives us the impression that
competitive interactions among them are highly plausible even if we
cannot infer their action by observational sampling alone.

\begin{blockquote}
How would these interactions compare to say priority effects in shaping
the community? I like the discussion of priority effects within the
colonization-competition trade off, but I'm less convinced that the
indirect facilitation interactions or intransitivity would actually be
taking place in a leaf, based on these data. Particularly given that
these interactions are being inferred from the pairwise interactions,
not directly assessed. I think caveats need to be added to the
discussion of these inferred interactions between 3 strains, as these
conclusions seem weaker to me than the conclusions about competitive
ability of the two clades or their potentially distinct life history
strategies.
\end{blockquote}

The \emph{in silico} finding is naturally less compelling than the
empirical one. However, illustrating that, in principle, Pf can modify
the outcome of itra-Ps interactions in potentially important ways is a
fairly straight-forward and plausible scenario to highlight, as we have
done with our simulation. We have added a good deal of language to the
Discussion to better contextualize the trio interaction study.

\begin{blockquote}
The introduction is set up very much in terms of ecological theory and I
think it could benefit from more examples or explanations related to
microbiology (specifics below). I found it somewhat hard to latch onto
at first, though the paper made more sense as I read more.
\end{blockquote}

We have modified the introduction to more readily appeal to a
microbe-focused audience, in order to facilitate the latching you refer
to.

\begin{blockquote}
It took me quite a while to figure out the methods for the competition
assays. I think the issue was some of the terminology used. This should
be clarified and it might help to have a figure in the SOM.
\end{blockquote}

We have added a schematic to the Supplemental Information (Fig. S\#)to
illustrate the design and have clarified the language around the
experimental description.

\begin{blockquote}
Specific points:
\end{blockquote}

\begin{blockquote}
The title is a little complicated for me, it requires thinking about a
lot of terminology. Since antibiosis isn't mentioned in the abstract as
a key point and doesn't come in until the results, maybe cut it from the
title? Competitive hierarchies as a function of bacterial life history
traits in a microbiome?
\end{blockquote}

We appreciate this suggestion and have modified the title accordingly.

\begin{blockquote}
Introduction:
\end{blockquote}

\begin{blockquote}
The initial framing (first sentence) of the article that we can use
plants to test microbiome theory fell flat for me, which may be in part
from my own bias. But much of the results were discussed with specific
relevance to plant microbiomes and these bacterial groups, so why not
just set it up to be about plant microbiomes?
\end{blockquote}

We have re-shaped the Introduction to avoid being shy about this paper
being purely about plant-associated microbes.

\begin{blockquote}
line 32 - I think this statement about PGPB might not make sense to a
wide audience. Maybe say something about plant microbiome manipulation
or probiotics already being used in agriculture?
\end{blockquote}

We have made mention of PGPB-like effects in more general terms in the
revision.

\begin{blockquote}
lines 36-38 - ``competition for shared resources and interference with
another species's ability to do so'' Interference with what? Competitive
ability? Or ability to access resources? It became clear to me what you
meant when I read the methods overview, but I think it would be helpful
to explain that you mean inhibition here.
\end{blockquote}

The reference here was in general terms to interference competition,
which can play out in a variety of ways. We have endeavored to make the
description of interference less vague in the revision.

\begin{blockquote}
lines 35-47 - I think this paragraph could benefit from a question or
example that is addressed by the paper (similar to lines 56-60 below),
to help orient the reader to what they are supposed to remember. As it
is the paragraph reads as very broad background but I'm not sure what
the paper is addressing, then later in the introduction the stated
questions of the study are mostly about direct vs.~indirect competition
not different types of competition. I think it doesn't help that the
next paragraph is about some taxa having a large impact, rather than
explicitly about inhibition, so I didn't pick up on this theme until
later.
\end{blockquote}

We hope that our Introduction re-writes have alleviated this concern.

\begin{blockquote}
lines 61-68 - I think this paragraph needs to be explained more or cut.
You could add an example to illustrate what an intransitive loop would
look like in a microbial community. I'm not sure this is a term that
more microbiology focused researchers would be familiar with. The
concept is explained better in the methods. However, given that you end
up finding that this type of loop seems to be uncommon in the data set,
and you don't discuss it much in the discussion, maybe it would be
better to leave it out of the intro?
\end{blockquote}

This is a good point. From our perspective, the intransivite loop has
received the majority of attention by microbe-focused researchers as an
interesting form of indirect interactions. We bring it up to point out
that this is by far not the only indirect interaction worth
understanding; far simpler motifs in communities may have a large effect
(such as the simple indirect facilitation avenue studied in more detail
in the paper). That said, we have clarified the language around this
topic so the reader is no longer confused about where the paper is
headed.

\begin{blockquote}
Methods:
\end{blockquote}

\begin{blockquote}
pairwise competition assays -
\end{blockquote}

\begin{blockquote}
I think stating that the resident was in a soft agar overlay would help
the microbiology types understand what was done here without having to
go into the SOM. I'm also not sure what ``above a negative control
spot'' means, are you referring to measuring the diameter of the spot?
\end{blockquote}

See comment above about our schematic SI figure. We have also clarified
the terminology used in the assay desription to be far more transparent.

\begin{blockquote}
I was particularly confused by what you meant by a ``megacolony.'' At
first I thought you meant the spots merging on the plate, but I think
you mean the growth in the spot itself or beyond the spot? Why not just
say spot? Or explain what you mean more, I'm not sure that megacolony is
a widely used term. My preference would be to not use it, it really
threw me.
\end{blockquote}

We have adopted the term ``spot'' throughout. Thanks for pointing this
out!

\begin{blockquote}
Is resident and invader the right terminology in this case, given that
they were placed on the plate at about the same time? I understand what
you're going for (I think), the resident is more abundant so it
theoretically has a competitive advantage against the other strain. But
resident and invader bring to mind different introduction times for me,
and the addition of quotes around the words doesn't help me understand
what you mean by them. Maybe something referencing relative abundance or
advantage instead? Or more clarification on why this particularly
experimental set up is relevant?
\end{blockquote}

Like most assay setups, ours was contrived. Our setup was not meant to
resemble any particular competitive context but was made up so as to
facilitate pairwise interactions at moderate scale. We have to call the
different strains something, at the end of the day; resident and invader
are poor choices, admittedly, so we have adopted simpler references to
the strain inoclated into the soft-agar overlay as the `bottom' strain,
and the strain spotted on top as the `top' strain.

\begin{blockquote}
I think the wording in the manuscript makes the scoring seem more
subjective than it may have been. Saying ``largely translucent'' in line
113 suggests to me that there were less translucent spots that were
scored differently than the largely translucent ones. Was there more
that went into this measure than the growth being contained to the spot
vs.~outside the spot? How easy were these to call? Again, I think some
example pictures might demonstrate how straightforward this scoring was
vs.~how much subjectivity went into it.
\end{blockquote}

We have pointed out examples of our different spot calls (SI fig
referenced in the text). In the end it was not difficult to discretize
the spot growth, since it was largely binary (with this intermediate
class being far less opaque than the full spots while showing more of a
``coffee stain'' shape where it was spotted). We have modified the
terminology to avoid giving the impression that this process was overly
subjective.

\begin{blockquote}
Results:
\end{blockquote}

\begin{blockquote}
What's up with strain 24A having missing data? Can you explain the
somewhere?
\end{blockquote}

I've looked back at this

\begin{blockquote}
Figure 2: I'm having a hard time telling the difference between the no
data grey and the lightest blue color. I think. 22C has no growth data?
\end{blockquote}

We have made the distinction more obvious in the revised figure.

\begin{blockquote}
lines 218 - Can you state what r and L are here, it's hard to remember
(although would be easier with the figure handy, it's just not in
reviewing the manuscript).
\end{blockquote}

We have added clarity throughout to reduce mental load of keeping things
straight.

\begin{blockquote}
Discussion:
\end{blockquote}

\begin{blockquote}
Line 303 - saying that P. fluorescens is presumed to be a soil dwelling
makes it sound like you are showing its importance on plants for the
first time, but I don't think that is accurate.
\end{blockquote}

This is not our intention, but that does not change the fact that many
references to Pf make more of its soil or rhizosphere association than
that is shows up in plant leave. We have clarified to ensure readers do
not think we are demonstrating this for the first time.

\begin{blockquote}
Referee \#3 (Comments to the Author):
\end{blockquote}

\begin{blockquote}
Whiteman and al.~performed large pairwise competition assays involving
different pseudomonas strains belonging to fluorescens and syringae
species. The found that P. fluorescens outcompete P. syringae in
competition on agar plate. Their manuscript is entitled ``Competitive
hierarchies, antibiosis, and the distribution of bacterial life history
traits in a microbiome'' ISMEJ-20-01631A.The results represent a
substantial advance in understanding the fine tuning of leaf microbiota
reflected by bacterial interactions. I very much enjoyed reading this
manuscript and have only a few criticisms, no more experiment are
required:
\end{blockquote}

We thank the reviewer for this feedback.

\begin{blockquote}
General comment.
\end{blockquote}

\begin{blockquote}
To my opinion, the data presented are potential of interest to
microbiology community. This manuscript is very well written. The
ecological notions and concepts are really well transposed from to
Microbiology.
\end{blockquote}

\begin{blockquote}
1- The title is not really enough informative and may lead to a
misunderstanding of what can be found in the text as the authors did not
study the leaf microbiota but competition between two pseudomonas
species that are part of the latter. I suggest to write a new one that
reflected the results provide inside the manuscript.
\end{blockquote}

This is a fair point. We did not study these strains `in' a microbiome,
but they are `from' a microbiome (notwithstanding the fact that similar
strains probably reside in a number of different putative `niches'). We
have modified the title in light of this as well as other comments from
revier \#2.

\begin{blockquote}
Methods :
\end{blockquote}

\begin{blockquote}
2- Calculation of the competitive indexes: In classical microbiology
approaches, the competitive index refers to the number of bacteria
recovered at ``T+ incubation time'' divided by the original number of
bacteria at t=0. In all the indexes presented in this study, no
temporality is included in the calculation and I think it is an
important point because the number of bacteria can slightly change in a
presence of a competitor, so based only on the colony forming
visualization or a presence of an inhibition halo might not be
sufficient.
\end{blockquote}

We have added language to distinguish our indexes from the classical
competitive index and instead highlight that these are summary
statistics of the interaction patterns observed in our particular assay.

\begin{blockquote}
3- Following this comment, as no starting point counting was performed,
it is difficult to know if the bacteria are confronted in 1:1 or
unbalanced ratios, that for sure change the fate of one of the two
competitor. How the authors can argue about this?
\end{blockquote}

We sought to control the starting densities as much as possible in our
work to be as close to parity in starting density. We have added text to
remind readers that performing these experiments using different ratios
would likely change the dynamics of the interactions. That said, we do
believe that waiting for as long as we did before scoring the outcomes
will help avoid any differences arising from starting density ratios
that would manifest transiently.

\begin{blockquote}
4- Curiosity question, Does the growth parameter be directly included in
the calculation of the competitive indexes? Could it be a way to
weight/correlate the presence of an halo to the growth of the bacteria
that produce inhibitory compound(s)?
\end{blockquote}

I'm not sure I understand the question. The growth parameter (measured
independently, in the plate reader) does not factor in to the
calculation of the competitive indexes that summarise the competitive
outcomes on the plates. If asking about whether the growth rate of Pf
strains correlates with whether the strain produced inhibitory
compound(s), then yes we have already included this calculation as part
of Fig. 2.

\begin{blockquote}
5- Several phylogenetic trees are represented in the figures, but
nothing is found about how they were build? In all the figures that
include trees, what are the values at each nodes of the tree? Bootstrp?
How many iterations ?
\end{blockquote}

The trees were built from data presented in Humphrey et al (2014), and
this fact is referenced in the Methods as well as the figure caption. We
have re-summarised those methods briefly in the revised manuscript.

\begin{blockquote}
Discussion:

6- The discussion should be tempered a bit, as the bacterial
competitions in this study was performed in a total artificial manners:
on agar plate, with an amount of bacteria that is not representative of
the real numbers found in plant leaf, not even the same ratio of the two
competitor on plant leaves, with a diversity of bacterial strains not
representative of the leaf microbiota (only Pseudomonas bacteria). I do
agree that this study is important for the microbial community
researcher, but here the authors provides competitions data only between
bacteria that belong to a same genus.
\end{blockquote}

We have added additional language to the Discussion to clarify that
these experiments merely demonstrate a range of \emph{potential}
interactions among strains that naturally occur in a native pant
microbiome. We also note that pseudomonads dominate (by a very large
number) the culturable and sequence-able microbiome from this and other
plants. Therefore, their ecological dominance in this sphere is a decent
motivator for focusing on them specifically in this (and other) works.
Along side of this point, we have also made sure to cite other
literature showing that non-pseudomonad lineages (e.g., Sphingomonas)
can generate competitive effects in plant microbiomes, so that the
impacts of microbial competition is by no means restricted to
\emph{Pseudomonas}.

\begin{blockquote}
7- Line 303-304: The authors performed these competition assays on
nutrient agar plate, the microbe interaction in a complex environment
such as plant leaves is totally different as two or several bacterial
strains can ``avoid'' competition by a different colonization pattern in
time and space. I suggest to be prudent by such extrapolation and to
tone it down.
\end{blockquote}

Again, this study was not meant to mimic a plant environment but to
demonstrate the distribution of possible interaction outcomes among
strains that commonly co-occur in a native plant microbiome. The issue
of how frequently strains come into contact with one another in nature
is an important one to consider. We have added language to clarify that
``the potential for the outcomes revealed here to impact microbial
community patterns in leaf microbiomes depends on the timing and scale
of co-colonization in actual living plants.''

\begin{blockquote}
8- Line 308-323: It is really hard to compare a lag time between
bacteria expose to an abiotic stress like carbon source or antibiotic to
a bacteria faced another one in competition. It depends of so many
parameters that are not well control in this study, like the initial
number of bacteria, the ratio of the competition, that are not really
representative to what is happening in the phyllosphere.
\end{blockquote}

Our growth experiments \emph{in vitro} reproducibly reveal intrinsic
differences in canonical features of the growth cycle among our focal
strians. As such, we expect these phenotypes, measured in this way, to
relate to some features of cellular physiology expressed under natural
conditions. We do agree that, in principle, bacterial colonizing plants
may express an entirely different suite of physiological patterns than
in culture such that differences in lag phase in culture will not map to
any commensurate trait expressed \emph{in planta}. This possibility
aside, we make clearer that this is merely an assumption we make when
considering how these interactions may play out in actual plants.

\begin{blockquote}
9- Line 329-332: Very interesting hypothesis.
\end{blockquote}

\begin{blockquote}
Minor comments: - Figure 2a : Hard to distinguish grey color with the
blue scale in the heat map, I suggest to change the color of the ``no
data'' category to a more distinct color.
\end{blockquote}

We have made this change.

\begin{blockquote}
\begin{itemize}
\tightlist
\item
  Line 333: the word ``exudate'' is more related to what a plant
  secreted, I suggest to replace exudate by ``metabolites'', or
  ``antimicrobial metabolites''
\end{itemize}
\end{blockquote}

We made clarified the language here.




\newpage
\singlespacing
\end{document}
