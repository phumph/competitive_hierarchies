\textbf{Life history variation, competition, and co-existence in a
bacterial phyllosphere community}

Parris T. Humphrey\textsuperscript{1,2*}, Trang N.
Nguyen\textsuperscript{1}, Noah K. Whiteman\textsuperscript{1,3}

\textsuperscript{1}Department of Ecology and Evolutionary Biology,
University of Arizona, 1041 E Lowell St., BioSciences West 310, Tucson,
AZ 85721.

\textsuperscript{2}Current address: Department of Organismic and
Evolutionary Biology, Harvard University, 26 Oxford Street, Cambridge,
MA 02138.

\textsuperscript{3}Current address: Department of Integrative Biology,
University of California, Berkeley, 3040 Valley Life Sciences Building,
Berkeley, CA 94720.

\textsuperscript{*}Corresponding author:
\href{mailto:phumphrey@fas.harvard.edu}{\nolinkurl{phumphrey@fas.harvard.edu}}

\textbf{Keywords\emph{.}} trade-off, interference competition,
exploitative competition, intransitivity, endophyte

\textbf{Author contributions.} PTH and NKW designed the study, PTH and
TNN collected the data, PTH wrote the manuscript with help from NKW.

\textbf{ABSTRACT}

\textbf{INTRODUCTION}

\textbf{METHODS}

\emph{Bacterial strains}

Of the 51 \emph{Pseudomonas} spp. strains isolated from bittercress and
described by Humphrey et al. (2014), we selected a set of 39 (26
\emph{P. syringae}, 14 \emph{P. fluorescens}) that represent the extent
of observed diversity. The laboratory strain \emph{P. syringae} pv.
maculicola str. ES4326 (hereafter Psm4326) was used as a reference owing
to its phylogenetic similarity to strains isolated from bittercress and
its extensive characterization in the laboratory as a pathogen of
\emph{Arabidopsis thaliana} \{Cui:2005dn, Cui:2002gp, Groen:2013bt,
Groen:2015bv\}.

\emph{Competition assays}

Pairwise competition assays were conducted on 1\% agar MM plates (100 mm
diameter) onto which we overlaid 4 ml of 0.5\% soft agar MM containing a
bacterial suspension of each ``resident'' strain inoculated at
5x10\textsuperscript{5}•ml\textsuperscript{-1} while soft agar was still
molten (\textasciitilde{}42°C). Suspensions of each of the 40
``invader'' strains were then spotted at the same concentration in 4 µL
aliquots spaced every 0.5 cm in parallel rows using an 8-channel
pipettor. Plates were incubated face up for 12 h, followed by face down
incubation at 28 °C for 10 days. ``Megacolony'' spots were scored by
hand for growth on days 1, 3, 5, 8, and 10. Data used for the following
analyses are from day 10, by which time all interactions dynamics had
reached equilibrium.

We scored growth of each invader as `0' for no visible growth of the
invader above a negative control spot containing MM alone, `0.5' for a
largely translucent megacolony which reflected a definite presence of
growth but which was relatively suppressed and confined to the
megacolony margin, and `1' for obvious and robust megacolony growth.
Examples of each can be seen in Fig. S3. We scored inhibition
interactions as the presence of a zone of clearance (halo) ≥ 1 mm
surrounding the extent of the invader megacolony (Fig. S3). Inhibition
interactions were ultimately scored as `0' or `1' regardless of the
spatial extent of the halo, although variation in halo width was
recorded. We also scored any morphological variation among megacolonies
for particular strains, and later we relate such variation for
particular strains to the phylogenetic position of their competitors
(see \emph{Analyzing the distribution of competitive outcomes}, below).

\emph{Calculating indexes of competitiveness}

Each strain was assayed under 40 different conditions both as resident
strain and as invader, comprising an interaction network with 1600
entries (including self vs. self). One version of the interaction
network represents the outcome of resource competition and details the
extent of growth of each invader, while the other captures the
presence/absence of inhibitory interactions indicated by zones of
clearance in the resident population. For resource competitions, we
calculate the invasiveness (\emph{C\textsubscript{o}}) and defense
capacity (i.e. territoriality) (\emph{C\textsubscript{d}}) of each
strain. \emph{C\textsubscript{o}} for strain \emph{i} was calculated as
\(C_{o,i} = \frac{1}{n_{\text{ij}}}\sum_{i \neq j}^{n}x_{\text{ij}}\),
where \(x_{\text{ij}} \in \{ 0,0.5,1\}\) and \(n_{\text{ij}}\) is the
total number of scored interactions for each strain as the invader with
all non-self resident strains. \emph{C\textsubscript{o }}is thus the
expected value of growth attained by each strain as the invader across
the population of residents. \emph{C\textsubscript{d}} was calculated
similarly except the focal strain \emph{j} is in the resident state,
\(x_{\text{ji}} \in \{ 0,0.5,1\}\) is as before but has a subscript
reversal and indicates the degree to which the resident prevented the
growth of each invader \emph{i}, and \(n_{\text{ji}}\) totals the number
of interactions occurring between each focal resident and its non-self
invaders. \emph{C\textsubscript{d}} can thus be interpreted as the
expected value of growth each resident strain can prevent among the
population of invaders assayed. We calculated an overall exploitative
competition index (\emph{C\textsubscript{w}}) was calculated for each
strain as \(C_{\text{w.i}} = C_{o,i} - {(1 - C}_{d,i})\), where --1 ≤
\emph{\emph{C\textsubscript{w}}} ≤ 1. These extremes represent absolute
competitive inferiority (--1), where a strain failed to prevent any
growth of any invader and similarly failed to invade any other strain,
to absolute competitive dominance (1), where a strain fully invaded all
residents and fully prevented growth of all invaders.

We calculated \emph{C\textsubscript{t}} and \emph{C\textsubscript{r}}
based on the interaction matrix for interference competition, where
\emph{C\textsubscript{t}} is the proportion of successful invasions
(i.e. given growth of 0.5 or above) that also resulted in halo
formation, indicative of inhibition of the resident.
\emph{C\textsubscript{r}} for a strain is the proportion of contests
with all invading inhibitor strains (i.e. all strains with
\emph{C\textsubscript{t}} \textgreater{} 0) that failed to result in
halo formation, which we took as evidence of resistance. An overall
interference competition index (\emph{I\textsubscript{w}}) was
calculated for each strain as \(I_{w,i} = C_{t,i} - {(1 - C}_{r,i})\),
where --1 ≤ \emph{I\textsubscript{w}} ≤ 1 (as in
\emph{C\textsubscript{w}}, above), which is equal to the aggressiveness
index (\emph{AI}) of Vestigian et al. \{*Vetsigian:2011je\}.

\emph{In vitro growth assays}

Strains were re-streaked from --80 °C stocks in 50\% glycerol onto
King's B (KB) agar plates +10 mM MgSO\textsubscript{4} and incubated at
28 °C for 3 days. Bittercress isolates had undergone only one prior
cycle of isolation--growth--freezing since initial isolate on KB plates
from surface-sterilized leaf homogenates from bittercress
\{Humphrey:2014ga\}. Single colonies were picked and inoculated into 1
mL minimal media at pH 5.6 \{MM; 10 mM fructose, 10 mM mannitol, 50 mM
KPO\textsubscript{4}, 7.6 mM
(NH\textsubscript{4})\textsubscript{2}SO\textsubscript{4}, and 1.7 mM
MgCl\textsubscript{2}; Mudgett:1999tu, Barrett:2011fo\} and grown
overnight in a shaking incubator (250 rpm) at 28 °C. MM at pH 5.6 has
been shown to induce the expression of the type-III secretion system
(T3SS) in a diversity of \emph{Pseudomonas} spp. \{Huynh:1989ux\}, in
contrast to KB, which results in negligible T3SS expression. T3SS
expression was important for maximizing the potential relevance of our
in vitro assay environments to those of plants, in which T3SS expression
is expected. Each 1 mL overnight culture was spun down for 3 m at 3,000
x \emph{g} and the supernatant was replaced with 500 µL fresh MM. The
density of each culture was adjusted to OD\textsubscript{600} = 0.2
prior to 1:100 dilution into a total of 180 µL MM inside the wells of
sterile polystyrene 96-well plates (Falcon). Each 96-well plate was
covered with BreathEasy® (Sigma \#Z380059) optically clear,
gas-permeable plastic tape and incubated for 60 hr in a BioTek 600 plate
reader in which OD\textsubscript{600} measurements were taken every 5 m
with continuous orbital shaking. Identical growth assays were performed
on separate days in duplicate.

\emph{Life history estimates}

We used R package \emph{grofit} \{Kahm:2010vv\} to fit smoothed
functions to the bacterial growth data. Curve fits generated using
logistic, Richards, Gompertz, or modified Gompertz equations failed to
produce estimates with \emph{r} ≥ 0.5 and we therefore used a
non-parametric locally-weighted smoothing (LOWESS) function to estimate
the following growth curve parameters: maximum growth rate
\emph{r\textsubscript{m}}, lag phase \emph{L}, and maximum yield
\emph{K}. Lag phase represents the length of time (min) prior to
initiation of exponential growth, while \emph{K} is the maximum
OD\textsubscript{600} attained during 60 h of growth. Curves for long
lag-phased strains never leveled off (Fig. S1, e.g. strain 17A); in
these cases, \emph{K} was set as the final OD\textsubscript{600}. When
growth trajectories exhibited multiple exponential phases (diauxic
shifts), \emph{r\textsubscript{m}} was estimated during the initial
exponential phase (e.g. strain 20A; Fig. S1).

To examine life history correlations, we calculated Pearson's \emph{r}
(correlation coefficients) between all pairs of growth and competitive
trait measurement, and statistical significance was assessed as
\emph{p}\textless{}0.05 after Benjamini--Hochberg false discovery rate
correction implemented in R package \emph{psych}
\{psychProceduresfo:2012tl\}. To further uncover fundamental axes of
trait covariances, we conducted principle components analysis (PCA)
using the matrix of mean-centered and scaled competitive indexes and
growth parameters for all strains (40 x 9 matrix) as input using
function \emph{prcomp} in the R base package \{RAlanguageanden:2012wf\}.
Finally, we constructed linear multiple regression models to estimate
the contribution of \emph{r\textsubscript{m}}, \emph{L}, and \emph{K} to
variation among \emph{P. syringae} and \emph{P. fluorescens} strains in
each of the overall competitive indexes \emph{C\textsubscript{w}} and
\emph{I\textsubscript{w}.}

\emph{Analyzing the distribution of competitive outcomes}

We determined when the outcomes of all pairwise interactions between
strains \emph{i} and \emph{j} (\emph{i ≠ j}) took following forms:
reciprocal invasibility (RI), where strains \emph{i} and \emph{j} invade
each other; reciprocal non-invasibility (RNI), where strains \emph{i}
and \emph{j} cannot invade each other; and asymmetric (AS), where strain
\emph{i} invades strain \emph{j} but \emph{j} cannot invade \emph{i}. We
constructed binomial generalized linear models (GLMs) in R with the
canonical logit link function to estimate the relationship between the
probability of RI, RNI, or AS versus genetic distance as well as trait
distance between strains \emph{i} and \emph{j} as predictors. Genetic
distance (\emph{D\textsubscript{G}}) was calculated as the pairwise raw
nucleotide distance between 2690 bp of sequence comprised of four
partial housekeeping gene sequences previously generated for each strain
from Humphrey et al. \{*Humphrey:2014ga\}. Orthologous sequences from
the genome of Psm4326 were derived from its published genome sequence
\{Baltrus:2011ci RefSeq ID NZ\_AEAK00000000.1\}. Euclidean distances
between each growth trait (\emph{r\textsubscript{m}}, \emph{L}, and
\emph{K}) for all pairs of strains were measured as
\(D_{\text{ij}} = \sqrt{{(x_{i} - x_{j})}^{2}}\). We first examined a
binomial model for each outcome type using \emph{D\textsubscript{G}} as
the only fixed effect, and then computed models including each growth
trait, which took the form
\(\text{logit}\left( P(y_{\text{ij}}|x_{\text{ij}}) \right)\ \sim\ \beta_{0} + \beta_{d}x_{d} + \beta_{r_{m}}x_{r_{m}} + \beta_{L}x_{L} + \beta_{K}x_{K}\).
To test for genetic correlations in trait values, we ran Mantel tests
between pairs of trait and genetic distance matrixes in R using package
\emph{vegan}. Test statistics were compared with those generated from
1000 matrix permutations \{veganCommunityEco:2012uw\}.

We noted instances where megacolony morphology differed between strain
pairings for particular isolates (e.g. \emph{P. fluorescens} str.
RM43A), and we compared the incidence of each discrete phenotype to the
phylogenetic position of the competitor strains using the genetic data
described above \{data from Humphrey:2014ga\}. To test the significance
of a phylogenetic correlation between the phylogenetic position of
competitor strains and the induced megacolony morphology of the focal
strains, we conducted a permutation analysis of variance (perMANOVA)
using the \emph{adonis} function of the R \emph{vegan} package with
10000 permutations.

\emph{Estimating indirect interactions from the pair-wise network}

We assembled all possible combinations of strain trios and evaluated
whether their patterns of interactions fulfilled the criteria for
facilitation described in the Introduction. Specifically, we calculated
the net first order linkage effects on each strain serving as the focal
strain in the presence of each other as the ``competitor'' strain, where
the interaction between the two is mediated by a nearby third strain.
Briefly, facilitation can occur by strain A releasing strain C from
inhibition from B (where A also has to be resistant to B's inhibitors),
or from resource competition from the superior competitor B. This
analysis remains agnostic to mechanism, but merely calculates the
proportion of conditions under which facilitation of an otherwise
less-fit competitor is expected to arise. We also determined the
prevalence of strain trios resulting in R--P--S intransitivity as well
as synergistic inhibition, where strain C is both out-competed by B and
inhibited by A, to which the B strain is resistant. For each strain, we
calculated the net effect of antagonistic vs. facilitative indirect
interactions across all possible trios and compare this to underlying
fitness metrics derived from the pair-wise interaction network.

We calculated the magnitude of such opportunity costs of being
non-resistant to quantify the predicted dependence of the indirect cost
of susceptibility and the indirect benefit of resistance. We compared
these two variables to their underlying dependence on relative
competitiveness in terms of resource use (Cw). Strong competitors lose
more by being sensitive, because they would have already won most
resource contests. In contrast, weaker resource competitors have much to
gain by being resistant, but little to lose: the relative improvement in
fitness increases dramatically as more contests are won owing to their
increased resistance. We modeled how the costs and benefits of
susceptibility and resistance depend on underlying resource
competitiveness by simulating so-called first-order linkage indirect
effects, where the focal interaction is impacted by a third associate
strain.

Paradoxically, these two effects cancel themselves out such that the
benefit of resistance is uniform. But does it scale symmetrically? But,
relatively speaking, greater gains accrue to weaker competitors because
the ∆N is a larger fraction of N\textsuperscript{*}. So this trivially
reduces to some y\textasciitilde{}1/x situation in relative terms. But
in absolute terms, not clear how much facilitation will actually improve
long-term prospects in the community. Overall fitness rank in the
community would change by some amount, which would boost residence
times. This could buy time for evolutionary rescue to take place, which
is the converse of how evolutionary rescue typically is considered (the
mirror image).
